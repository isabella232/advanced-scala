
\documentclass{beamer}

\usepackage[utf8]{inputenc}

\usepackage{verbatim}

\usepackage{listings}

% "define" Scala
\lstdefinelanguage{scala}{
  morekeywords={abstract,case,catch,class,def,%
    do,else,extends,false,final,finally,%
    for,if,implicit,import,match,mixin,%
    new,null,object,override,package,%
    private,protected,requires,return,sealed,%
    super,this,throw,trait,true,try,%
    type,val,var,while,with,yield},
  otherkeywords={=>,<-,<\%,<:,>:,\#,@},
  sensitive=true,
  morecomment=[l]{//},
  morecomment=[n]{/*}{*/},
  morestring=[b]",
  morestring=[b]',
  morestring=[b]"""
}

\usepackage{color}
\definecolor{dkgreen}{rgb}{0,0.6,0}
\definecolor{gray}{rgb}{0.5,0.5,0.5}
\definecolor{mauve}{rgb}{0.58,0,0.82}
\definecolor{codebackground}{rgb}{1,1,0.7} % Default settings for code listings

\lstset{
  %frame=tb,
  language=scala,
  aboveskip=3mm,
  belowskip=3mm,
  showstringspaces=false,
  columns=flexible,
  basicstyle={\small\ttfamily},
  numbers=none,
  numberstyle=\tiny\color{gray},
  keywordstyle=\color{blue},
  commentstyle=\color{dkgreen},
  stringstyle=\color{mauve},
  numbersep=5pt,
  frame=none,
  breaklines=true,
  breakatwhitespace=true,
  tabsize=2,
  backgroundcolor=\color{codebackground}
}


\usetheme{default}
%\useoutertheme[footline=authorinstitutetitle]{miniframes}
\useoutertheme{infolines}
\usecolortheme{beaver}
\setbeamertemplate{navigation symbols}{}
\setbeamertemplate{footline}[page number]
\setbeamercolor{itemize item}{fg=red} % bullet color 1st level
\setbeamercolor{itemize subitem}{fg=red} % bullet color 2nd level
\setbeamercolor{itemize subsubitem}{fg=red} % bullet color 3rd level
\setbeamercolor{block title}{fg=black} % set block title fg color to black

\usepackage{tikz}
\usetikzlibrary{arrows}
\tikzstyle{block}=[draw opacity=0.7,line width=1.4cm]

%%%%%%%%%%%%%%%%%%%%%%%%%%%%%%%%%%%%%%%%%%%%%%%%%%%%%%%%%%%%%%%%%%%%%%%%%%%%%%%%%%%%%%%%%%%%%%%%%%%%%%%%%%%%%%%
%%%%%%%%%%%%%%%%%%%%%%%%%%%%%%%%%%%%%%%%%%%%%%%%%%%%%%%%%%%%%%%%%%%%%%%%%%%%%%%%%%%%%%%%%%%%%%%%%%%%%%%%%%%%%%%
%%%%%%%%%%%%%%%%%%%%%%%%%%%%%%%%%%%%%%%%%%%%%%%%%%%%%%%%%%%%%%%%%%%%%%%%%%%%%%%%%%%%%%%%%%%%%%%%%%%%%%%%%%%%%%%

\title{Ein Griff in die Scala-Trickkiste}
\author{Heiko Seeberger, Roman Roelofsen}
\institute{WeigleWilczek}
\date{JAX, Mainz, 2. Mai 2011}
\begin{document}


\begin{frame}
\titlepage
\end{frame}


\begin{frame}[fragile]{}
\begin{center}
	\textbf{Builder Pattern}
\end{center}
\end{frame}

%%%%%%%%%%%%%%%%%%%%%%%%%%%%%%%%%%%%%%%%%%%%%%%%%%%%%%%%%%%%%%%%%%%%%%%%%%%%%%%%%%

\begin{frame}[fragile]{Builder Pattern}
\begin{itemize}
	\item API Design zum Setzen mehrere Parameter
	\item setter-Methode-Ansatz
	\begin{lstlisting}
		val config = new Config
		config.setA(1)
		config.enableLog()
		config.setB(2)
		val created = config.create()
	\end{lstlisting}
	\item Builder-Pattern-Ansatz
	\begin{lstlisting}
		val created =
		  new Config().setA(1).setB(2).enableLog().create()
	\end{lstlisting}	
\end{itemize}
\end{frame}

%%%%%%%%%%%%%%%%%%%%%%%%%%%%%%%%%%%%%%%%%%%%%%%%%%%%%%%%%%%%%%%%%%%%%%%%%%%%%%%%%%

\begin{frame}[fragile]{Builder Pattern}
\begin{itemize}
	\item Methodenaufrufe geben i.d.R. \textbf{this} zurück...
	\item ... oder ein Objekt welches sich die Konfiguration "merkt"
	\item Manche Konfiguration stehen eventuell erst nach dem Setzen bestimmer Werte zur Verfügung
	\begin{lstlisting}
		val created = new Config().enableLog().withDbLog().create()
	\end{lstlisting}
\end{itemize}
\end{frame}  
  
%%%%%%%%%%%%%%%%%%%%%%%%%%%%%%%%%%%%%%%%%%%%%%%%%%%%%%%%%%%%%%%%%%%%%%%%%%%%%%%%%%

\begin{frame}[fragile]{Builder Pattern - Implementierung}
\begin{itemize}
	\item Definition
		\begin{lstlisting}
		class ABad {
		  var number = 1
  
		  def double(): ABad = {
		    number *= 2
		    this
		  }
		  def triple(): ABad = {
		    number = number * 3
		    this
		  }
		}
		\end{lstlisting}
	\item Verwendung
		\begin{lstlisting}
		println(new ABad().double().triple().number)
		\end{lstlisting}
\end{itemize}
\end{frame}

%%%%%%%%%%%%%%%%%%%%%%%%%%%%%%%%%%%%%%%%%%%%%%%%%%%%%%%%%%%%%%%%%%%%%%%%%%%%%%%%%%

\begin{frame}[fragile]{Builder Pattern - Implementierung}
\begin{itemize}
	\item Definition
		\begin{lstlisting}
		class BBad1 extends ABad {
		  def random(): BBad1 = {
		    number *= Random.nextInt
		    this
		  }
		}
		\end{lstlisting}
	\item Verwendung
		\begin{lstlisting}
		println(new BBad1().random().number)
		println(new BBad1().random().triple().number)
		\end{lstlisting}
	\item Problem
		\begin{lstlisting}
		println(new BBad1().double().triple().random().number)
		\end{lstlisting}
\end{itemize}
\end{frame}

%%%%%%%%%%%%%%%%%%%%%%%%%%%%%%%%%%%%%%%%%%%%%%%%%%%%%%%%%%%%%%%%%%%%%%%%%%%%%%%%%%

\begin{frame}[fragile]{Builder Pattern - Implementierung}
\begin{itemize}
	\item Lösung (wie in Java)
		\begin{lstlisting}
		class BBad2 extends ABad {
		  override def double(): BBad2 = {
		    super.double()
		    this
		  }

		  override def triple(): BBad2 = {
		    super.triple()
		    this
		  }

		  def random(): BBad2 = {
		    number *= Random.nextInt
		    this
		  }
		}
		\end{lstlisting}
\end{itemize}
\end{frame}

%%%%%%%%%%%%%%%%%%%%%%%%%%%%%%%%%%%%%%%%%%%%%%%%%%%%%%%%%%%%%%%%%%%%%%%%%%%%%%%%%%

\begin{frame}[fragile]{Builder Pattern - Implementierung}
\begin{itemize}
	\item Lösung (wie in Java)
		\begin{itemize}
			\item Alle Methodendeklarationen müssen wiederholt werden
			\item Lösung ist sehr fragil, da Änderungen in der Superklasse (z.B. neue Methode) die Kette "unterbrechen" können
			\item Alle Subklassen müssen informiert und nachgezogen werden
		\end{itemize}
\end{itemize}
\end{frame}

%%%%%%%%%%%%%%%%%%%%%%%%%%%%%%%%%%%%%%%%%%%%%%%%%%%%%%%%%%%%%%%%%%%%%%%%%%%%%%%%%%

\begin{frame}[fragile]{Builder Pattern - Implementierung}
\begin{itemize}
	\item Lösung mit Scala this.type
		\begin{lstlisting}
		class AGood {
		  var number = 1
		  def double(): this.type = {
		    number *= 2
		    this
		  }
		  def triple(): this.type = {
		    number = number * 3
		    this
		  }
		}
		\end{lstlisting}
\end{itemize}
\end{frame}

%%%%%%%%%%%%%%%%%%%%%%%%%%%%%%%%%%%%%%%%%%%%%%%%%%%%%%%%%%%%%%%%%%%%%%%%%%%%%%%%%%

\begin{frame}[fragile]{Builder Pattern - Implementierung}
\begin{itemize}
	\item Lösung mit Scala this.type
		\begin{lstlisting}
		class BGood extends AGood {
		  def random(): this.type = {
		    number *= Random.nextInt
		    this
		  }
		}
		\end{lstlisting}
	\item Verwendung
		\begin{lstlisting}
		println(new BGood().double().triple().random().number)
		\end{lstlisting}
\end{itemize}

\end{frame}

%%%%%%%%%%%%%%%%%%%%%%%%%%%%%%%%%%%%%%%%%%%%%%%%%%%%%%%%%%%%%%%%%%%%%%%%%%%%%%%%%%

\begin{frame}[fragile]{Stackable Modifications}
\begin{itemize}
	\item Logging Klassen
		\begin{itemize}
			\item Basis-Interface: \verb!Log!
			\item Implementierungen: \verb!NormalLog!, \verb!DBLog!
		\end{itemize}
		\pause
	\item Modifizierungen
		\begin{itemize}
			\item \verb!ImportantLog! (Text + '!!!')
			\item \verb!SuperImportantLog! (Großbuchstaben)
		\end{itemize}
		\pause
	\item Ziel
		\begin{itemize}
			\item Log-Implementierungen sollen beliebig mit den Modifizierungen erweitert werden können
			\item Modifizierungen müssen sich kombinieren lassen
			\item Je nach Komplexität kann die Reihenfolge wichtig sein
		\end{itemize}
\end{itemize}
\end{frame}

%%%%%%%%%%%%%%%%%%%%%%%%%%%%%%%%%%%%%%%%%%%%%%%%%%%%%%%%%%%%%%%%%%%%%%%%%%%%%%%%%%

\begin{frame}[fragile]{Stackable Modifications - Log Klassen}
\begin{lstlisting}
abstract class Log {
  def log(msg: String): Unit
}

class ConsoleLog extends Log {
  def log(msg: String) {
    println("LOG: " + msg)
  }
}

class DBLog extends Log {
  def log(msg: String) {
    Database.save(msg)
  }
}
\end{lstlisting}
\end{frame}

%%%%%%%%%%%%%%%%%%%%%%%%%%%%%%%%%%%%%%%%%%%%%%%%%%%%%%%%%%%%%%%%%%%%%%%%%%%%%%%%%%

\begin{frame}[fragile]{Stackable Modifications - Modifizierungen}
\begin{lstlisting}
trait ImportantLog extends Log {
  abstract override def log(msg: String) {
    super.log(msg +  "!!!")
  }
}

trait SuperImportantLog extends Log {
  abstract override def log(msg: String) {
    super.log(msg.toUpperCase)
  }
}
\end{lstlisting}
\end{frame}

%%%%%%%%%%%%%%%%%%%%%%%%%%%%%%%%%%%%%%%%%%%%%%%%%%%%%%%%%%%%%%%%%%%%%%%%%%%%%%%%%%

\begin{frame}[fragile]{Stackable Modifications - Verwendung}
\begin{lstlisting}
class AppLog extends ConsoleLog with ImportantLog with SuperImportantLog
val log1 = new AppLog
log1.log("A log message")

val log2 = new ConsoleLog with ImportantLog
log2.log("Another message")

val log3 = new DBLog with SuperImportantLog
log3.log("Another message")

val log4 = new DBLog with ImportantLog with SuperImportantLog
log4.log("Another message")
\end{lstlisting}
\end{frame}

%%%%%%%%%%%%%%%%%%%%%%%%%%%%%%%%%%%%%%%%%%%%%%%%%%%%%%%%%%%%%%%%%%%%%%%%%%%%%%%%%%

\begin{frame}[fragile]{Kontrollstrukturen abstrahieren}
\center \textbf{Aufgabe} \linebreak	
(1) Resource öffnen (2) Resource verwenden \linebreak (3) Resource schließen (4) Fehler abfangen
\center \textbf{Java}
\begin{lstlisting}[language=java]
InputStream is = null;
try {
  is = new FileInputStream(new File("..."));
  is.read();
} catch (Exception e) {
  // ...
} finally {
  try {
    is.close();
  } catch (Exception e1) {
    // ...
  }
}
\end{lstlisting}
\end{frame}  

%%%%%%%%%%%%%%%%%%%%%%%%%%%%%%%%%%%%%%%%%%%%%%%%%%%%%%%%%%%%%%%%%%%%%%%%%%%%%%%%%%

\begin{frame}[fragile]{Kontrollstrukturen abstrahieren}
\center \textbf{Aufgabe} \linebreak	
(1) Resource öffnen (2) Resource verwenden \linebreak (3) Resource schließen (4) Fehler abfangen
\center \textbf{Scala}
\begin{lstlisting}
withResource(new FileInputStream(new File("..."))) { is =>
  // ...
}
\end{lstlisting}
\end{frame}  

%%%%%%%%%%%%%%%%%%%%%%%%%%%%%%%%%%%%%%%%%%%%%%%%%%%%%%%%%%%%%%%%%%%%%%%%%%%%%%%%%%

\begin{frame}[fragile]{Kontrollstrukturen abstrahieren}
\center \textbf{Aufgabe} \linebreak	
(1) Resource öffnen (2) Resource verwenden \linebreak (3) Resource schließen (4) Fehler abfangen
\center \textbf{Scala Implementierung}
\begin{lstlisting}
def withResource[A <: { def close() }](res: A)(block: A => Any) {
  try {
    block(res)
  } catch {
    case e: Throwable => throw e
  } finally {
    try {
      res.close()
    } catch {
      case e: Throwable => throw e
    }
  }
}
\end{lstlisting}
\end{frame}  

%%%%%%%%%%%%%%%%%%%%%%%%%%%%%%%%%%%%%%%%%%%%%%%%%%%%%%%%%%%%%%%%%%%%%%%%%%%%%%%%%%

\begin{frame}[fragile]{Kontrollstrukturen abstrahieren}
\begin{itemize}
	\item Java-Code verwendet oft Exceptions als gültigen Rückgabewert
	\item Automatische Exception / Option Konvertierung im Scala-Kontext daher sinnvoll
\end{itemize}
\begin{lstlisting}
import scala.util.control.Exception._

val saveParse = catching(classOf[NumberFormatException])
val numberOption = saveParse opt { "123b".toInt }   

numberOption match {
  case Some(n) => n
  case None => 0
}
\end{lstlisting}
\end{frame}  

 
\end{document}
